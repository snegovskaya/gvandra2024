\section{Общегеографическая и туристическая характеристика района}

\subsection{Географическое положение и туристские особенности района}
Тут всякие красивые слова про всякое. Могу накатать сам, Можешь  сама, Даш. 


\subsection{Варианты подъезда и отъезда}
Подъезд осуществлён на поезде 033М Москва--Владикавказ до станции Минеральные Воды (прибытие в 03:40) Стоимость проезда на август 2024 г. составляла 7800~\faRub, купе (обратно – 4700~\faRub, плацкарт). От Минеральных Вод до аула Верхний Учкулан (время в пути 4 часа) добирались на трансфере, заказанном через Саракуева Бориса (89289503868, 89298843175,  \href{mailto: bezonec@list.ru}{bezonec@list.ru}). Стоимость трансфера трансфера туда составила 18000~\faRub, обратно (от поляны Азау) — 15000~\faRub. Стоимости доставки забросок в т/б Глобус и а/л Узункол составили 4000~\faRub  и 6000~\faRub.

\subsection{Аварийные выходы из маршрута и его запасные варианты}
\textbf{Аварийными выходами} с маршрута являлись:
\begin{itemize}
	\item На первом этапе: спуск к т/б <<Глобус>>;
	\item На втором этапе: спуск к а/л <<Узункол>>;
	\item На третьем этапе: спуск к погранзаставе <<Хурзук>>
\end{itemize}


\textbf{Запасными вариантами} маршрута являлись:
\begin{itemize}
	\item Замена пер. Уллу-Кёль Восточный (1А$^\star$, 3050) на пер. \textbf{Уллу-Кёль Нижний (н/к, 2933)};
	\item На втором этапе: спуск к а/л <<Узункол>>;
	\item На третьем этапе: спуск к погранзаставе <<Хурзук>>
\end{itemize}
- на первом этапе: спуск к т/б «Глобус»;
- на втором этапе: спуск к а/л «Узункол», спуск к пос. Хурзук.
Запасной вариант: вместо пер. Кертмели-Учкуланичи Южный (1А, 3248 м) - пер. Ножу Сев. (1А, 3327 м); вместо пер. Уллукель Восточный (1А*, 3050 м) - пер. Уллу-Кель Нижний (н/к, 2933м); вместо Кичкинекол Малый (1А, 3206 м) - пер. Доломиты Южный (1А, 3350 м).

\subsection{Характеристика средств передвижения, особенности погодных условий}

\subsection{Расположение пограничных зон, заповедников, порядок получения пропусков, дислокация ПСО, медучреждений и другие полезные данные}

\subsection{Перечень наиболее интересных природных и исторических объектов, занятий на маршруте}
\newpage