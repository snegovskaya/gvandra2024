\subsection{Джалпаккол Северный} 
Перевал Джалпаккол Северный должен был представлять из себя второй по сложности перевал на маршруте после Уллу-Кёля Восточного. Руководитель проходил этот перевал в 2018 году, и в том походе, по отзывам участников, с точки зрения он был явно лидирующем. Характерной особенностью перевала является наличие под его взлётом ледовой линзы, которую приходится огибать, проходя под гребнем из разрушенных скал. Этот участок камнеопасен, и в нём заключается первая опасность при прохождении данного перевала. Вторая опасность заключается в том, что есть возможность сорваться и укатиться на несколько десятков метров вниз по ледовой линзе: в 2018 году руководитель стал свиделем этого. Особенно велика вероятность этого при прохождении перевала без кошек, поэтому, как уже было сказано выше, от участников кошки требовались в обязательном порядке. Наконец, третья ключевая особенность перевала --- пятиметровый участок разрушенных скал перед выходом на седловину, который приходится преодолевать свободным лазанием. Для некоторых участников, которые неуверенно себя чувствую на скалах, этот участок мог стать большой проблемой. 

Перевал ориентирован с северо-запада на юго-восток, что несколько смягчает условия его прохождения с точки зрения раскисания снега и камнеопасности, однако по-прежнему рассматривался руководителем как один из самых сложных. Альтернативой данному перевалу мог бы служить Джалпаккол Южный (1А), расположенный, соответственно, южнее, но поскольку Джалпаккол Северный, с одной стороны, разнообразнее и богаче с точки зрения техники, а с другой стороны, у руководителя был опыт его прохождения, то выбор был сделан однозначно в его пользу. 

При прохождении перевала группа опиралась на воспоминания руководителя и отчёт Королёва Андрея \cite{Korolyov2018}, а также на отчёты...