\subsection{пер. Хотютау} 
Перевал Хотютау (1А, 3550 м) расположен на Юго-Западном ребре Эльбруса, ориентирован с запада на восток и связывает долину одного из левых притоков р. Уллу-Кам (которая, в свою очередь, является правым притоком -- родоначальником Кубани) с ледовыми полями Эльбруса (нижней плоской частью лед. Большой Азау). Этот перевал также проходился руководителем в 2018 году, и, по воспоминаниям, не вызвал сложностей. На тот момент, с учётом времени года, перевальный взлёт с обеих сторон был осыпным: со стороны р. Уллу-Кам довольно занудным, но технически простым, а со стороны Эльбруса до момента выхода на ледник --- коротким и простым. Прохождение же ледовых полей Эльбруса хотя и могло сорваться в случае, если бы ледник оказался закрытым, но вероятность этого в нашем случае, во второй половине августа, была низкой. В связи с этим ориентироваться на месте предполагалось в первую очередь на воспоминания руководителя и отчёт Королёва Андрея (кон.~июля -- нач.~августа 2018) \cite{Korolyov2018} соответственно, а также на отчёты Александра Севидова (вторая половина июля 2021) \cite{Sevidov2021} и Дмитрия Брунарского (кон.~июля -- нач.~августа 2023) \cite{Brunarsky2023}.